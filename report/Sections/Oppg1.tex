\section{Task 1}
We wish to look at reaction-diffusion equations, which have the form
$$ u_t = \mu u_{xx} + f(u)$$
where $\mu$ is a positive constant. We also assume that the reaction term, $f(u)$ is non-stiff, such that explicit methods can be used to solve $u_t = f(u)$.
We discretize the domain in both space and time, such that $x_m = x_0 + m*h, \quad t_n = t_0 + n*k$.
A scheme based on forward and backward Euler, together with a central difference in space, could be:
$$ \frac{1}{k}\nabla_tU_{m}^{n+1}=\frac{\mu}{h^2}\delta_x^2U_{m}^{n+1}+f(U_{m}^{n} )$$
which can be rewritten to 
$$U_{m}^{n+1} = U_{m}^{n} + r (U_{m+1}^{n+1}-2U_{m}^{n+1}+U_{m-1}^{n+1}) + kf(U_{m}^{n}), \quad r = \mu\frac{k}{h^2}$$
Time dependent PDEs should be solved using implicit methods. This requires solving a linear system for each step. We wish to avoid this, by using an implicit method for the diffusion term and an explicit method for the reaction term.
This could be solved be using the scheme
\begin{align}
  \label{eq:scheme_implicit_eq}
    U_{m}^{*} =& U_{m}^{n} +\frac{r}{2}(\delta_x^2 U_{m}^{*} + \delta_x^2 U_{m}^{n} ) + kf(U_{m}^{n}) \\
  \label{eq:scheme_explicit_eq}
    U_{m}^{n+1} =& U_{m}^{*} + \frac{k}{2}(f(U_{m}^{*}) - f(U_{m}^{n}))
\end{align}

\subsection{Stability analysis}

Define the matrix \( S \) as
\[
  S = 
  \begin{bmatrix}
  -2 & 1 &  &  & \\
   1& -2 & 1 &  & \\
   & \ddots & \ddots & \ddots & \\
   &  & 1 & -2 & 1\\
   &  &  & 1 & -2\\
  \end{bmatrix}
\]
and define the vector \( U^n = [U_1^n, U_2^n, \dots, U_M^n]^T \)
by combining the function value at the \( M \) inner points into a vector.
To make the matrix-vector product well defined, let \( S \) be of dimensions
\( M \times M \).

Rewriting equation \ref{eq:scheme_implicit_eq} as a matrix-vector equation
and collecting the \( U^* \)-s on the left hand side yields:
yields
\begin{equation}
  (I - \frac{r}{2}S)U^* = (I + kaI + \frac{r}{2}S) U^n
\end{equation}

Since the inverse of \( I - \frac{r}{2}S \) exists, we substitute for \( U^* \)
in equation \ref{eq:scheme_explicit_eq}:
\begin{align}
  U^{n+1} &= U^n + \frac{ka}{2}\left(U^*  - U^n\right) \\
          &= \left(1 + \frac{ka}{2}\right) U^* - \frac{ka}{2}U^n \\
          &= \left(1 + \frac{ka}{2}\right) {\left[I - \frac{r}{2}S\right]}^{-1} \left(I + kaI + \frac{r}{2}S\right) U^n - \frac{ka}{2}U^n \\
          &= {\left[I - \frac{r}{2}S\right]}^{-1} \left(\left(1 + \frac{ka}{2}\right)\left(I + kaI + \frac{r}{2}S\right) - \frac{ka}{2}\left(I-\frac{r}{2}S\right)\right) U^n \\
          &= {\left[I - \frac{r}{2}S\right]}^{-1} \left( \left(1+ka+\frac{1}{2}(ka)^2\right)I + \frac{1}{2}\left(1+ka\right)rS\right) U^n \\
          &= C U^n
\end{align}
where we let 
\begin{equation}
    C = {\left[I - \frac{r}{2}S\right]}^{-1} \left( \left(1+ka+\frac{1}{2}(ka)^2\right)I + \frac{1}{2}\left(1+ka\right)rS\right)
  \end{equation}

\begin{lemma}
  \label{lemma:S_diagonalizable}
  The matrix \( S = \text{tridiag}(1, -2, 1) \in \text{Mat}_{M, M}(\mathbb{R}) \)
  is diagonalizable by orthogonal matrices \( P \): \( S = P \Lambda P^T \).
  The eigenvalues of \( S \) are
  \( \lambda_m = - \sin^2 \phi_m, \phi_m = \frac{m \pi}{2(M+1)}, m = 1, \dots, M\).
\end{lemma}
\begin{proof}
    duuuh.
\end{proof}

By using lemma \ref{lemma:S_diagonalizable} we arrive at a diagonalization for \( C \):
\begin{align}
  C &= {\left[I - \frac{r}{2}S\right]}^{-1} \left( \left(1+ka+\frac{1}{2}(ka)^2\right)I + \frac{1}{2}\left(1+ka\right)rS\right) \\
    &= {\left[PP^T - \frac{r}{2}P\Lambda P^T\right]}^{-1} \left( \left(1+ka+\frac{1}{2}(ka)^2\right)PP^T + \frac{1}{2}\left(1+ka\right)r P\Lambda P^T\right) \\
    &= P{\left[I - \frac{r}{2}\Lambda \right]}^{-1} \left( \left(1+ka+\frac{1}{2}(ka)^2\right)I + \frac{1}{2}\left(1+ka\right)r \Lambda \right) P^T \\
    &= P \Delta P^T
\end{align}
where
\begin{equation}
    \Delta = {\left[I - \frac{r}{2}\Lambda \right]}^{-1} \left( \left(1+ka+\frac{1}{2}(ka)^2\right)I + \frac{1}{2}\left(1+ka\right)r \Lambda \right)
\end{equation}
The eigenvalues for \( C \) are found on the diagonal of \( \Delta \):
\begin{equation}
  \Delta_m =
      \frac{\left(1+ka+\frac{1}{2}(ka)^2\right) + \frac{1}{2}\left(1+ka\right)r \lambda_m}
      {1 - \frac{r}{2} \lambda_m}
\end{equation}

The condition \( \rho(C) \le 1 + \nu k \) is
both necessary and sufficient for stability since \( C \) is symmetric.
A bound on \( \rho(C) = \max_{m} \lvert \Delta_m \rvert \) is found:
\begin{align}
  \left\lvert \Delta_m \right\rvert &= 
\left\lvert \frac{\left(1+ka+\frac{1}{2}(ka)^2\right) + \frac{1}{2}\left(1+ka\right)r \lambda_m}
{1 - \frac{r}{2} \lambda_m} \right\rvert \\
 &= 
 \left\lvert \left(1+ka\right)\frac{1+\frac{1}{2}r \lambda_m}{1 - \frac{1}{2}r \lambda_m} + \frac{1}{2}(ka)^2 \frac{1}{1 - \frac{1}{2}r \lambda_m} \right\rvert \\
 &\le \left\lvert \left(1+ka\right)\frac{1+\frac{1}{2}r \lambda_m}{1 - \frac{1}{2}r \lambda_m} \right\rvert + \left\lvert\frac{1}{2}(ka)^2 \frac{1}{1 - \frac{1}{2}r \lambda_m} \right\rvert \\
 &= 
 \left\lvert \left(1+ka\right)\frac{1+\frac{1}{2}r \lambda_m}{1 - \frac{1}{2}r \lambda_m} + \frac{1}{2}(ka)^2 \frac{1}{1 - \frac{1}{2}r \lambda_m} \right\rvert \\
 &\le \left\lvert \left(1+ka\right) \right\rvert \left\lvert \frac{1+\frac{1}{2}r \lambda_m}{1 - \frac{1}{2}r \lambda_m} \right\rvert + \frac{1}{2}(ka)^2 \left\lvert\frac{1}{1 - \frac{1}{2}r \lambda_m} \right\rvert \\
 \intertext{Note that \( -1 < \lambda_m < 0  \) for all \( m = 1, \dots, M \), so we can simplify further:}
 &\le 1 + (\lvert a \rvert + \frac{1}{2}k a^2 )k \\
 &\le 1 + (\lvert a \rvert + \frac{1}{2}T a^2)k \\
 &= 1 + \nu k
\end{align}

We summarize our discussion in a theorem.

\begin{theorem}
    The scheme is unconditionally stable.
\end{theorem}

